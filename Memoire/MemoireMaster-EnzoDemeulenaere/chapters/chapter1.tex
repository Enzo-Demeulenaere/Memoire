%%%%%%%%%%%%%%%%%%%%%%%%%%%%%%%%%%%% Chapter Template

\chapter{Introduction} 	% Main chapter title
\label{Chapter1} 		% For referencing the chapter elsewhere, usage \ref{Chapter1}

%%%%%%%%%%%%%%%%%%%%%%%%%%%%%%%%%%%%
Ce document vise à guider l'étudiant dans l'élaboration de son mémoire de Master.

L'auteur doit tenir compte des règles générales suivantes lors de la préparation du document:
\begin{itemize}
    \item Il est essentiel de réfléchir, avant d'écrire, à la substance de ce qui est destiné à être transmis;
    \item Vous devez organiser le texte en évitant sa division excessive en sujets qui sont censés tomber sous le thème principal de la section à laquelle ils appartiennent. Une spécialisation excessive peut révéler un manque de connaissances et/ou de réflexion. En ce sens, vous devez, avant de commencer à écrire, travailler à affiner l'organisation du texte afin d'éviter qu'il y ait plus de 2 niveaux de ``profondeur'' dans chaque section (respectivement, sous-section et sous-sous-sous-section). Notez que le nombre de chapitres, sections et sous-sections de ce document n'est pas contraignant ni même indicatif, il ne sert qu'à ses fins;
    \item Le document doit être rédigé en français ou en anglais dans un style approprié (éviter le ton familier, les lieux communs et les mots à la mode) et correct d'un point de vue grammatical (que ce soit d'un point de vue syntaxique ou sémantique);
    \item Soyez particulièrement prudent avec l'utilisation des adjectifs (ils conduisent facilement à l'exagération), des adverbes (rien, ou presque rien, ajoutent-ils) et des signes de ponctuation (surtout l'utilisation correcte des virgules);
    \item Le style adopté pour l'écriture doit être conforme aux exigences d'un travail scientifique trouvé dans les publications imprimées;
    \item D'une manière générique, vous devriez utiliser la 3ème personne du singulier (éventuellement du pluriel), sauf pour les endroits où c'est clairement déplacé, par exemple, dans la section de remerciements;
    \item Utilisez le style \textit{italic} chaque fois que des termes sont utilisés dans des langues autres que la langue adoptée dans le rapport;
    \item L'utilisation d'acronymes implique que la première fois qu'ils sont utilisés, ils doivent être présentés en entier, en mettant entre parenthèses l'acronyme respectif qui sera utilisé. Cependant, il est toujours possible, plus tard dans le texte, et dans un souci de lisibilité, de répéter le sens de l'acronyme. Tous les acronymes doivent être présentés par ordre alphabétique dans la section ``Liste des acronymes'';
    \item L'utilisation correcte des unités, de leurs multiples et sous-multiples;
    \item Les images et les tableaux doivent, en principe, apparaître en haut ou en bas de la page. Les légendes apparaissent immédiatement après les chiffres et les listes. Dans le cas des tableaux, les légendes les précèdent;
    \item Toutes les figures, tableaux et autres listes doivent être mentionnés dans le texte afin qu'ils soient encadrés dans les idées transmises par l'auteur. Cette référence, en règle générale, doit être faite avant l'apparition de la figure, du tableau ou de la liste;
    \item Il doit indiquer tout au long du texte les références documentaires utilisées, notamment dans les citations (pures ou littérales), marquées de l'utilisation de guillemets, ainsi qu'en cas de réutilisation de graphiques, figures, tableaux, formules, etc., d’autres sources.
\end{itemize}

Plus précisément, dans ce premier chapitre obligatoire (``Introduction''), l'auteur doit:
\begin{itemize}
    \item contextualiser la proposition de travail au sein de l'entreprise, à partir d'un autre travail déjà réalisé, d'un point de vue scientifique et/ou technologique, etc.,
    \item présente clairement les objectifs qu'il se propose d'atteindre,
    \item décrivent succinctement, mais objectivement, la solution ou l'hypothèse recommandée,
    \item présente brièvement mais clairement les développements réalisés,
    \item identifie comment la solution trouvée a été validée et évaluée,
    \item décrivent l'organisation du document.
\end{itemize}

Sans nécessiter d'organisation particulière pour ce chapitre, 4 sections (avec le texte \gls{lipsum}) sont indiquées à titre d'exemple, qui peuvent être incorporées dans cette partie du document: Contextualisation, Description du projet, Calendrier et Organisation du rapport.
%%%%%%%%%%%%%%%%%%%%%%%%%%%%%%%%%%%% SECTION 1

\section{Contextualisation}
\label{sec:Ch1.1}

\lipsum[1] % inserts fake text, to be removed in your dissertation

%%%%%%%%%%%%%%%%%%%%%%%%%%%%%%%%%%%% SECTION 2

\section{Description du projet}
\label{sec:Ch1.2}

\lipsum[2]

%%%%%%%%%%%%%%%%%%%%%%%%%%%%%%%%%%%% SUBSECTION 1

\subsection{Objectifs}
\label{sub:Ch1.2.1}

\lipsum[2]

%%%%%%%%%%%%%%%%%%%%%%%%%%%%%%%%%%%% SECTION 3

\section{Calendrier}
\label{sec:Ch1.3}

\lipsum[1]
% you could use the pgfgantt package: https://ctan.org/pkg/pgfgantt


%%%%%%%%%%%%%%%%%%%%%%%%%%%%%%%%%%%% SECTION 4

\section{Organisation du rapport}
\label{sec:Ch1.4}

\lipsum[2]

