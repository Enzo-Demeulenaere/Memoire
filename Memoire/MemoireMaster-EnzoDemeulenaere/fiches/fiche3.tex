% Fiche nº3

\chapter{Fiche nº3} % Main appendix title
\label{app:Fiche3} % For referencing this appendix elsewhere, use \ref{app:Fiche3}

%%%%%%%%%%%%%%%%%%%%%%%%%%%%%%%%%%%%%%%%%%%%%%%%%%%%%%%%%%%%%%%%%%%%%%%%%%%%%%%%%%
\section{Description de l'article}

\paragraph{Titre de l'article~:}
\paragraph{Lien de l'article~:}
\paragraph{Liste des auteurs~:}
\paragraph{Affiliation des auteurs~:}
\paragraph{Nom de la conférence / revue~:}
\paragraph{Classification de la conférence / revue~:}
\paragraph{Nombre de citations de l'article (quelle source ?)~:}



%%%%%%%%%%%%%%%%%%%%%%%%%%%%%%%%%%%%%%%%%%%%%%%%%%%%%%%%%%%%%%%%%%%%%%%%%%%%%%%%%%
\section{Synthèse de l'article}

\paragraph{Problématique}

Les recherches sur l'apprentissage en VR semblent se concentrer sur une comparaison des médias par lesquels passe l'apprentissage,
plutôt que sur les facteurs humains qui guident l'apprentissage.

\paragraph{Pistes possibles (pointés par les auteurs)}

La négligeance des ressentis de l'utilisateur dans son environnement d'apprentissage représente un obstacle dans les objectifs d'apprentissage

\paragraph{Question de recherche}

Quels sont les réels impacts de l'immersion et de l'intéractivité en VR sur les capacités d'apprentissage.

\paragraph{Démarche adoptée}

Regarder les impacts isolés des notions d'intéractivité et d'immersion,
tester le framework CAMIL

\paragraph{Implémentation de la démarche}

Expérience où une leçon sera instruite à travers différents média à l'immersion et à l'interactivité diverses (vidéo,PC, VR-video, VR)
(Ici l'immersion est surtout visuelle)
180 participants adultes, 153 données ont été analysées.

Creation d'un musée virtuel sur le sujet de maladies virologiques
2 versions: une visite pré-enregistrée (interactivité basse), une visite libre (interactivité haute)
Tester empiriquement les hypothèses présentées par le framework CAMIL en utilisant une modélisation d'équation structurelle (SEM)


\paragraph{Les résultats}
\lipsum[1]

Hypothesis of CAMIL on VR learning
"the path from VR features to learning outcomes was mediated by non-cognitive outcomes such as motivation, presence, and usability"


both interactivity and immersion were shown to impact agency, physical presence, and embodied learning positively

Although interactivity still influenced agency and embodied learning when lessons were highly immersive, the impact was not as strong as when the lesson was less immersive. Supposedly, being visually immersed is such a powerful experience in itself that interactivity provides less added value; under conditions of low visual immersion, however, interactivity is shown to its full advantage

No effects of interactivity or immersion on intrinsic motivation, self-efficacy, extraneous cognitive load interaction, or learning were found.