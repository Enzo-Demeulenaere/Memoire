%----------------------------------------------------------------------------------------
%	ABSTRACT PAGES
%----------------------------------------------------------------------------------------

% IMPORTANT NOTE: the abstract must always be written in two languages. If the report
% is written in Portuguese you have selected 'portuguese' as the language in the document class.
% Therefore, the portuguese version of the abstract must come first, so write it in the
% below area denoted by 'MAIN LANGUAGE ABSTRACT'. The english version follows in the
% 'SECOND LANGUAGE ABSTRACT' section.
% If the report is written in English, first will come the abstract in English
% ('MAIN LANGUAGE ABSTRACT') and then in Portuguese ('SECOND LANGUAGE ABSTRACT').

\begin{abstract}
%%%%%%%%%%%%%%%%%%%%%%%%%%%%%% MAIN LANGUAGE ABSTRACT %%%%%%%%%%%%%%%%%%%%%%%%%%%%%%%%%%

Ici, le résumé de tous les travaux réalisés doit être présenté. Cette section ne doit pas dépasser une page.

Vous devez contextualiser le problème que vous souhaitez résoudre ou l'hypothèse que vous allez formuler, essayer de mettre en évidence les avantages et les inconvénients (le cas échéant) de la solution trouvée, ainsi que la manière dont la solution/hypothèse a été validée. Dans ce dernier point, il doit se référer aux développements réalisés, et à la manière dont il a validé (conformité) et évalué (performance) la solution trouvée.

Le document doit toujours contenir deux versions du résumé: une première dans la langue du texte principal et la seconde dans une autre langue. Ce \textit{template} suppose que les deux langues considérées sont toujours le portugais et l'anglais, donc la classe placera les en-têtes respectifs en fonction de la langue sélectionnée dans les options de classe dans le fichier \file{main.tex}.

%----------------------------------------------------------------------------------------

\vspace*{10mm} 
\noindent
\textbf{\keywordslabel}: Liste, séparés par des virgules, des mots, des phrases ou des acronymes clés dans le cadre des travaux décrits dans ce texte.

%%%%%%%%%%%%%%%%%%%%%%%%% END OF THE MAIN LANGUAGE ABSTRACT %%%%%%%%%%%%%%%%%%%%%%%%%%%%%%
\end{abstract}
\begin{secondlangabstract}
%%%%%%%%%%%%%%%%%%%%%%%%%%%%%% SECOND LANGUAGE ABSTRACT %%%%%%%%%%%%%%%%%%%%%%%%%%%%%%%%%%

The summary of all the developed work should be presented here. This section should not exceed one page.

Start the abstract with the contextualization of the problem you intend to solve or the hypothesis you will formulate. Try to highlight the advantages and disadvantages (if any) of the solution found, as well as the way in which the solution/hypothesis was validated. In this last point, you should refer to the developments made, and to the way you validated (compliance) and evaluated (performance) the solution found.

The document must always contain two versions of the abstract: a first in the language of the main text and the second one in another language. This template assumes that the two languages are always Portuguese and English, therefore, the class will place the correct section headers according to the language selected in the class options in the \file{main.tex} file.


%----------------------------------------------------------------------------------------

\vspace*{10mm} 
\noindent
\textbf{\keywordslabel}: Comma separated list of words, phrases, or key acronyms within the scope of your developed work. 

%%%%%%%%%%%%%%%%%%%%%%%%%% END OF THE SECOND LANGUAGE ABSTRACT %%%%%%%%%%%%%%%%%%%%%%%%%%%%%
\end{secondlangabstract}

